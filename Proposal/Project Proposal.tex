% Options for packages loaded elsewhere
\PassOptionsToPackage{unicode}{hyperref}
\PassOptionsToPackage{hyphens}{url}
%
\documentclass[
]{article}
\usepackage{amsmath,amssymb}
\usepackage{iftex}
\ifPDFTeX
  \usepackage[T1]{fontenc}
  \usepackage[utf8]{inputenc}
  \usepackage{textcomp} % provide euro and other symbols
\else % if luatex or xetex
  \usepackage{unicode-math} % this also loads fontspec
  \defaultfontfeatures{Scale=MatchLowercase}
  \defaultfontfeatures[\rmfamily]{Ligatures=TeX,Scale=1}
\fi
\usepackage{lmodern}
\ifPDFTeX\else
  % xetex/luatex font selection
\fi
% Use upquote if available, for straight quotes in verbatim environments
\IfFileExists{upquote.sty}{\usepackage{upquote}}{}
\IfFileExists{microtype.sty}{% use microtype if available
  \usepackage[]{microtype}
  \UseMicrotypeSet[protrusion]{basicmath} % disable protrusion for tt fonts
}{}
\makeatletter
\@ifundefined{KOMAClassName}{% if non-KOMA class
  \IfFileExists{parskip.sty}{%
    \usepackage{parskip}
  }{% else
    \setlength{\parindent}{0pt}
    \setlength{\parskip}{6pt plus 2pt minus 1pt}}
}{% if KOMA class
  \KOMAoptions{parskip=half}}
\makeatother
\usepackage{xcolor}
\usepackage[margin=1in]{geometry}
\usepackage{graphicx}
\makeatletter
\def\maxwidth{\ifdim\Gin@nat@width>\linewidth\linewidth\else\Gin@nat@width\fi}
\def\maxheight{\ifdim\Gin@nat@height>\textheight\textheight\else\Gin@nat@height\fi}
\makeatother
% Scale images if necessary, so that they will not overflow the page
% margins by default, and it is still possible to overwrite the defaults
% using explicit options in \includegraphics[width, height, ...]{}
\setkeys{Gin}{width=\maxwidth,height=\maxheight,keepaspectratio}
% Set default figure placement to htbp
\makeatletter
\def\fps@figure{htbp}
\makeatother
\setlength{\emergencystretch}{3em} % prevent overfull lines
\providecommand{\tightlist}{%
  \setlength{\itemsep}{0pt}\setlength{\parskip}{0pt}}
\setcounter{secnumdepth}{-\maxdimen} % remove section numbering
\ifLuaTeX
  \usepackage{selnolig}  % disable illegal ligatures
\fi
\IfFileExists{bookmark.sty}{\usepackage{bookmark}}{\usepackage{hyperref}}
\IfFileExists{xurl.sty}{\usepackage{xurl}}{} % add URL line breaks if available
\urlstyle{same}
\hypersetup{
  pdftitle={Backflipping Bicycles (Project Proposal)},
  pdfauthor={Joseph Camacho},
  hidelinks,
  pdfcreator={LaTeX via pandoc}}

\title{Backflipping Bicycles (Project Proposal)}
\author{Joseph Camacho}
\date{March 22, 2023}

\begin{document}
\maketitle

\hypertarget{main-idea}{%
\subsection{Main Idea}\label{main-idea}}

I want to simulate a bicyclist doing backflips off a ramp. A "success"
is a robot that manages to do a complete backflip and land on its wheels
without crashing.

I plan to start with a basic model that models the bicycle as two
wheels, a seat, and handlebars and the bicyclist as a torso with two
arms:\\
\begin{center}
\includegraphics[height=7cm]{/home/joseph/mit-classes/Excalidraw/Project Proposal 2023-03-22 22.35.41.excalidraw.png}\\
\end{center}
and add more details later (if I can successfully simulate this first).

The bicycist will have two controls:
\begin{enumerate}
\tightlist
\item
  He can use the pedals, accelerating the bicycle forwards.
\item
  He can pull up on the handlebar, leading to an equal and opposite push
  down on the seat.
\end{enumerate}
These are the only controls he will have-\/-it\textquotesingle s a
very underactuated system.  This is an interesting problem because at first glance it might seem impossible to do a back flip with only these controls, but humans have been able to successfully do so anyways.

\hypertarget{topics-related-to-class}{%
\subsection{Topics Related to Class}\label{topics-related-to-class}}

\begin{enumerate}
\tightlist
\item
  Underactuated robotics
\item
  Stability theory, probably using Lyapunov analysis (can I prove that
  the robot will be able to balance, for example?)
\item
  Trajectory optimization
\end{enumerate}

\hypertarget{method}{%
\subsection{Method}\label{method}}

I plan to use Drake to do the simulation. I\textquotesingle m not yet
sure which trajectory optimization methods I\textquotesingle ll use,
though I plan to use something simple like an iterative linear quadratic
regulator (as done in the homework exercise 10.3) if possible.

\hypertarget{related-papers}{%
\subsection{Related Papers}\label{related-papers}}

I found a couple papers on bicycle simulation and learning stunts, but
none that did backflipping in particular, and they mostly used
reinforcement learning. Two of the more relevant papers I found were
\url{https://ieeexplore.ieee.org/stamp/stamp.jsp?tp=\&arnumber=8571685}
and \url{https://dl.acm.org/doi/pdf/10.1145/2601097.2601121}.

\end{document}
